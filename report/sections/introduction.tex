\section{Introduction}

This report provides an in-depth overview of data visualization, covering fundamental concepts and the optimal use-cases for various chart types, while also considering the needs of different user demographics. The science of visual perception is examined, emphasizing its role in effective design. Additionally, we explore the relationship between data and design principles, stressing the importance of consistency between data preprocessing decisions and design choices. The report presents the 'Grammar of Graphics,' which is a framework that simplifies the design process. It then explores various techniques for assessing the efficacy of data visualizations. The overall objective of this report is to offer a comprehensive comprehension of data visualization by merging its theoretical, practical, and evaluative components.

\subsection{Introduction to the Gapminder Dataset}

The Gapminder dataset provides a comprehensive examination of numerous socio-economic and health-related aspects across countries and selected timeframes. The variables consist of country name, year, population, continent, life expectancy, and GDP per capita \cite{GapminderWorld}. The following is a concise explanation of the dataset variables:

\begin{itemize}
    \item \textbf{Country}: The name of the country.
    \item \textbf{Year}: The year the data was recorded.
    \item \textbf{Population}: The total population of the country for the corresponding year.
    \item \textbf{Continent}: The continent to which the country belongs.
    \item \textbf{Life Expectancy}: The average life expectancy in years for people born in that year.
    \item \textbf{GDP per Capita}: The Gross Domestic Product per capita, which serves as an indicator of the economic performance of a country.
\end{itemize}